% \documentclass[11pt]{article}
\documentclass[UTF8]{ctexbook}

\usepackage{graphicx} % Required for inserting images
\setlength{\parindent}{0pt}
\usepackage{enumitem}
\usepackage[utf8]{inputenc} 
\usepackage[T1]{fontenc}
\usepackage[english]{babel}
\usepackage{lipsum}
\usepackage{hyperref}


% 使日期显示为英文格式
% \CTEXoptions[today=old]

\usepackage[left=1.06cm,top=1.7cm,right=1.06cm,bottom=0.49cm]{geometry}

\begin{document}
\begin{center}
    \textbf{Chen (Charlie) Fang}\\ 
    \hrulefill
\end{center}

\begin{center}
    Haidian District, Beijing, 100084, China \textbullet \ \href{mailto:fangc23@mails.tsinghua.edu.cn}{fangc23@mails.tsinghua.edu.cn} \textbullet \ \href{https://charlie-pku.github.io//}{personal page}
\end{center}

\vspace{0.5pt}

\begin{center}
    \textbf{EDUCATION}
\end{center}
\textbf{Tsinghua University} \hfill Beijing, China

% Degree, Concentration. GPA [Note: GPA is Optional] \hfill Graduation Date Thesis [Note: Optional]

% Relevant Coursework: [Note: Optional. Awards and honors can also be listed here.]

\quad \textbf{Ph.D. Student, Economics}, School of Economics and Management (SEM) \hfill 2023 - 2028 (Expected)


\vspace{12pt}

\textbf{Peking University}  \hfill Beijing, China

\quad \textbf{B.S., Environmental Sciences}, College of Environmental Sciences and Engineering (CESE) \hfill	2019 - 2023

\quad \textbf{B.A., Economics (Double Degree)}, National School of Development (NSD) \hfill 2021 - 2023

\vspace{12pt}

% \textbf{High School Name} \hfill	City, State

% [Note: May include GPA, SAT/ACT scores, or academic honors an employer may want to know] \hfill Graduation Date

% \vspace{12pt}

\begin{center}
    \textbf{RESEARCH INTEREST}
\end{center}

Climate Change Economics, Environmental and Resource Economics, Urban and Spatial Economics

\vspace{12pt}

\begin{center}
    \textbf{PUBLICATIONS}
\end{center}


\begin{itemize}
    \item ``Negative emission technology is key to decarbonizing China's cement industry'' with Ming Ren, Teng Ma, Xiaorui Liu, Chaoyi Guo, Silu Zhang, Ziqiao Zhou, Yanlei Zhu, Hancheng Dai and Chen Huang. \textbf{Applied Energy} (2023), 329, 120254.

    \item ``Global land-use and sustainability implications of enhanced bioenergy import of China'' with Yazhen Wu, Andre Deppermann, Petr Havlík, Stefan Frank, Ming Ren, Hao Zhao, Lin Ma, Qi Chen, Hancheng Dai. \textbf{Applied Energy} (2023), 336, 120769.

    \item ``Enhanced food system efficiency is the key to China's 2060 carbon neutrality target'' with Ming Ren, Chen Huang, Yazhen Wu, Andre Deppermann, Stefan Frank, Petr Havlík, Yuyao Zhu, Xiaotian Ma, Yong Liu, Hao Zhao, Jinfeng Chang, Lin Ma, Zhaohai Bai, Shasha Xu and Hancheng Dai. \textbf{Nature Food} (2023), 1-13.
\end{itemize}

\vspace{12pt}

\begin{center}
    \textbf{WORKS IN-PROGRESS}
\end{center}


\begin{itemize}
    \item ``The Effect of Key Pollutant Discharge Firms on Local Second-hand Housing Prices in Guangdong Province'' with Yana Jin
    \item ``An Integrated Assessment of Provincial Economic Damages from Climate Change in China'' with Hantang Peng, Tianpeng Wang, Da Zhang and Xiliang Zhang
    \item ``Forest Autonomy and Forest Carbon Sink Potential: Evidence from Northeast China'' with Tingyu Cui and Yushuai Zhang
    \item ``Research on Transnational Mobility of Researchers based on Multi-Agent Modeling: A case study of Artificial Intelligence'' with Wenjie Chen and Chao Min
    \item ``Trade Ban and Environmental Pollution: A Policy Evaluation on Trade Ban of Foreign Garbage'' with Wenhui Yang
\end{itemize}

\vspace{12pt}


\begin{center}
    \textbf{TEACHING EXPERIENCE}
\end{center}


\begin{itemize}
    \item Principles of Economics (2) (Undergraduate, TA for Prof. Feng Dong) \hfill 2024-Spring
    \item Principles of Economics (1) (Undergraduate, TA for Prof. Xiaohan Zhong and Yingyi Qian) \hfill 2023-Autumn
    \item Intermediate Microeconomics (Undergraduate, TA for Prof. Huayu Xu) \hfill 2023-Spring
\end{itemize}

\vspace{12pt}

\begin{center}
    \textbf{RESEARCH ASSISTANT}
\end{center}


\begin{itemize}
    \item ``The Economic Impact of Carbon Peak and Neutrality in China and China's Green and Low-carbon Transition'', Institute of Energy, Environment and Economy, THU (PI: Da Zhang and Xiliang Zhang) \hfill Aug. 2021 - now
    \item ``The Economics Analysis of Verified Carbon Standard (VCS)'', National School of Development, PKU (PI: Cong Peng, Xianling Long and Shilei Liu) \hfill Jan. 2023 - Aug. 2023
    \item ``the Low-Carbon Development of Institutional Investors `Collaborative Engagement' Enterprises in ESG Field'', Institute of Finance and Sustainability (PI: Jiayin Zhao) \hfill Jun. 2022 - Oct. 2022
    \item ``Update the policy report \textit{Applying the Growth Identification and Facilitation Framework to Nepal (UN CDP Background Paper No. 35)}'', Institute of New Structural Economics, PKU (PI: Jiajun Xu) \hfill Jul. 2022 - Sep. 2022

\end{itemize}

\vspace{12pt}

\begin{center}
    \textbf{AWARDS}
\end{center}


\begin{itemize}
    \item Environmental Alumni Scholarship \hfill 2022
    \item Award for Academic Excellents \hfill 2022
    \item Second Prize of the $13^{rd}$ National Undergraduate Mathematics Competition \hfill 2022
    \item Second Prize of the $30^{th}$ ``Challenge Cup'' competition of Peking University \hfill 2022
    \item Third Prize of the $2^{nd}$ ``SFLEP·VocabGo Cup'' National English Vocabulary Contest \hfill 2022

    \item Third-class Scholarship of Peking University  \hfill 2021
    \item Merit Student \hfill 2021
    \item Outstanding Winner of the $29^{th}$ ``Challenge Cup'' competition of Peking University  \hfill 2021
    \item Second Prize of the $7^{th}$ China National College Students Competition on Energy Economics \hfill 2021

    
    \item XIAOMI Scholarship \hfill 2020
    \item Award for Academic Excellents \hfill 2020

\end{itemize}

\vspace{12pt}

\begin{center}
    \textbf{SKILLS}
\end{center}


\textbf{Language:}   Chinese (native), English (Fluent) \\
\textbf{Programming:}   \LaTeX, Python, R, STATA, QGIS

% \textbf{Laboratory:} List scientific / research lab techniques or tools [If Applicable]

% \textbf{Interests:} List activities you enjoy that may spark interview conversation

\vspace{12pt}


\hfill \textbf{Last Updated}: \today

% \textbf{Organization} \hfill City, State (or Remote)
 
% \textbf{Position Title} \hfill Month Year – Month Year
% \begin{itemize}[noitemsep]
%     \item Beginning with your most recent position, describe your experience, skills, and resulting outcomes in bullet or paragraph form.
%     \item Begin each line with an action verb and include details that will help the reader understand your accomplishments, skills, knowledge, abilities, or achievements.
%     \item Quantify where possible.
%     \item Do not use personal pronouns; each line should be a phrase rather than a full sentence.
% \end{itemize}

% \vspace{12pt}

% \textbf{Organization} \hfill City, State (or Remote)

% \textbf{Position Title} \hfill Month Year – Month Year
% \begin{itemize}[noitemsep]
%     \item With your next-most recent position, describe your experience, skills, and resulting outcomes in bullet or paragraph form.
%     \item Begin each line with an action verb and include details that will help the reader understand your accomplishments, skills, knowledge, abilities, or achievements.
%     \item Quantify where possible.
%     \item Do not use personal pronouns; each line should be a phrase rather than a full sentence.
% \end{itemize}

% \begin{center}
%     \textbf{Leadership \& Activities}
% \end{center}

% \textbf{Organization}	\hfill City, State

% \textbf{Role} \hfill Month Year – Month Year
% \begin{itemize}[noitemsep]
%     \item This section can be formatted similarly to the Experience section, or you can omit descriptions for activities.
%     \item If this section is more relevant to the opportunity you are applying for, consider moving this above your Experience section.
% \end{itemize}


\end{document}
